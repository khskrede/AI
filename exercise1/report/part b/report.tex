\documentclass[11pt]{article}
\usepackage{graphicx}
\usepackage{url}
\usepackage[small]{caption2}
\renewcommand{\baselinestretch}{1.0}
\title{Project 1}
\author{Knut Halvor Skrede}
\setlength{\parindent}{0pt}
\setlength{\parskip}{2ex} 
\begin{document}
\maketitle
\clearpage

\section*{Part B)}

\subsection*{Problem representation}

\subsubsection*{Genotypes and phenotypes}

The genotype is represented as an array of $B*S$ boolean values (where $S$ is the number of available
soldiers and $B$ is the number of battles.) The genotype was chosen such that no genes would be "worth"
more soldiers than others.

A restriction is set on the genotype to contain exactly $S$ true values. The genotype is initialized 
such that each boolean of the genotype has a probability of $1/B$ chance of being true.

The phenotype is represented as an array of $B$ integers. These integers represent how many troops 
the colonel shows up with at each battle.

\subsubsection*{Crossover}

The crossover is done by finding all places where in the genotype array individual 1 is true and individual 2 is not.
And then finding all places where in the genotype array individual 2 is true and individual 1 is not. Then using this
information to randomly construct a bit-mask to be used in the crossover.

\subsubsection*{Mutation}

Mutation is done by selecting two random places in the genotype that has different values, and switching places.
This guarantees that there will still be $S$ troops.

\subsubsection*{Evaluation}

The evaluation is done exactly as proposed by the assignment.

\subsubsection*{Strategy entropy}

The average strategy entropy is calculated as proposed by the assignment and plotted along with the 
fitness plots.

\subsection*{EA settings}

The settings used for all ea runs:\\
\\
Size of child pool: 30\\
Size of adult pool: 30\\
number of generations: 200\\
mutation rate: 1\\
crossover rate: 0.9\\
selection protocol: Full generational replacement\\
selection strategy: Sigma scaling\\
elitism: 2\\

\subsection*{Table of strategies and results}

\begin{figure}[ht]

\begin{tabular}{c|c|c|c|c}
\hline\hline
B & Rf & Lf & Shifts between \\
\hline
5 & 1.0 & 0.0 &  Basicly no shifting\\
5 & 0.0 & 1.0 &  Basicly no shifting\\
5 & 1.0 & 1.0 &  Basicly no shifting\\
\hline
20 & 1.0 & 0.0 &  Basicly no shifting\\
20 & 0.0 & 1.0 &  Basicly no shifting\\
20 & 1.0 & 1.0 &  Basicly no shifting\\
\hline
5 & 0.0 & 0.0 & Basicly no shifting \\
8 & 0.0 & 0.0 & Basicly no shifting \\
11 & 0.0 & 0.0 & 2 basic groups \\
14 & 0.0 & 0.0 & 3 basic groups \\
17 & 0.0 & 0.0 &  A variety of groups\\
20 & 0.0 & 0.0 & A variety of groups\\
\hline
15 & 1.0 & 1.0 &  Basicly no shifting\\
15 & 0.7 & 0.0 &  Basicly no shifting\\
15 & 0.0 & 7.0 &  Basicly no shifting\\
15 & 0.5 & 0.5 &  Basicly no shifting\\
\hline
8 & 0.0 & 1.0 &  Basicly no shifting\\
8 & 1.0 & 0.0 &  Basicly no shifting\\
8 & 1.0 & 1.0 &  Basicly no shifting\\
\hline
\end{tabular}

\caption{Example of }
\label{fig:1}
\end{figure}

\subsection*{descriptions of the 3 signature cases}

All the runs resulted in converging results.
Don't have time to add fintess plots. The plots for
all runs are included in a zip file.
The plots are named {B-Rf-Lf}.png, they should correspond
to the table above.

\end{document}


